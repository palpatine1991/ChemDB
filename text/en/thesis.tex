%%% The main file. It contains definitions of basic parameters and includes all other parts.

%% Settings for single-side (simplex) printing
% Margins: left 40mm, right 25mm, top and bottom 25mm
% (but beware, LaTeX adds 1in implicitly)
\documentclass[12pt,a4paper]{report}
\setlength\textwidth{145mm}
\setlength\textheight{247mm}
\setlength\oddsidemargin{15mm}
\setlength\evensidemargin{15mm}
\setlength\topmargin{0mm}
\setlength\headsep{0mm}
\setlength\headheight{0mm}
% \openright makes the following text appear on a right-hand page
\let\openright=\clearpage

%% Settings for two-sided (duplex) printing
% \documentclass[12pt,a4paper,twoside,openright]{report}
% \setlength\textwidth{145mm}
% \setlength\textheight{247mm}
% \setlength\oddsidemargin{14.2mm}
% \setlength\evensidemargin{0mm}
% \setlength\topmargin{0mm}
% \setlength\headsep{0mm}
% \setlength\headheight{0mm}
% \let\openright=\cleardoublepage

%% Character encoding: usually latin2, cp1250 or utf8:
\usepackage[utf8]{inputenc}

%% Further useful packages (included in most LaTeX distributions)
\usepackage{amsmath}        % extensions for typesetting of math
\usepackage{amsfonts}       % math fonts
\usepackage{amsthm}         % theorems, definitions, etc.
\usepackage{bbding}         % various symbols (squares, asterisks, scissors, ...)
\usepackage{bm}             % boldface symbols (\bm)
\usepackage{graphicx}       % embedding of pictures
\usepackage{fancyvrb}       % improved verbatim environment
\usepackage{natbib}         % citation style AUTHOR (YEAR), or AUTHOR [NUMBER]
\usepackage[nottoc]{tocbibind} % makes sure that bibliography and the lists
			    % of figures/tables are included in the table
			    % of contents
\usepackage{dcolumn}        % improved alignment of table columns
\usepackage{booktabs}       % improved horizontal lines in tables
\usepackage{paralist}       % improved enumerate and itemize
\usepackage[table]{xcolor}  % typesetting in color
\usepackage{nameref}


\PassOptionsToPackage{hyphens}{url}

%%% Basic information on the thesis

% Thesis title in English (exactly as in the formal assignment)
\def\ThesisTitle{Comparison of Approaches for Querying of Chemical Compounds}

% Author of the thesis
\def\ThesisAuthor{Bc. Vojtěch Šípek}

% Year when the thesis is submitted
\def\YearSubmitted{2019}

% Name of the department or institute, where the work was officially assigned
% (according to the Organizational Structure of MFF UK in English,
% or a full name of a department outside MFF)
\def\Department{Department of Software Engineering}

% Is it a department (katedra), or an institute (ústav)?
\def\DeptType{Department}

% Thesis supervisor: name, surname and titles
\def\Supervisor{Doc. RNDr. Irena Holubová, Ph.D.}

% Supervisor's department (again according to Organizational structure of MFF)
\def\SupervisorsDepartment{Department of Software Engineering}

% Study programme and specialization
\def\StudyProgramme{Computer Science}
\def\StudyBranch{Software Systems}

% An optional dedication: you can thank whomever you wish (your supervisor,
% consultant, a person who lent the software, etc.)
\def\Dedication{%
I would like to take this opportunity to thank to the people who helped me during my studies and without whom I would not be able to finish this work. My supervisor Doc. RNDr. Irena Holubová, Ph.D. for great support during the whole time I has been writing this thesis. My family for the support during my whole studies. My brother, who helped me to submit this thesis while I was abroad. And my girlfriend for never-ending reminders to finish this work and not to give up.
}

% Abstract (recommended length around 80-200 words; this is not a copy of your thesis assignment!)
\def\Abstract{%
The purpose of this thesis is to perform an analysis of approaches to querying chemical databases and to validate or invalidate its results. Currently, there exists no work which would compare the performance and memory usage of the best performing approaches on the same data set. In this thesis, we address this lack of information and we create an un-biased benchmark of the most popular index building methods for subgraph querying of chemical databases. Also, we compare the results of such benchmark with the performance results of an SQL and a graph database.
}

% 3 to 5 keywords (recommended), each enclosed in curly braces
\def\Keywords{%
{Chemical database,} {Subgraph querying,} {Graph database,} {Subgraph isomorphism}
}

%% The hyperref package for clickable links in PDF and also for storing
%% metadata to PDF (including the table of contents).
\usepackage[pdftex,unicode]{hyperref}   % Must follow all other packages
\hypersetup{breaklinks=true}
\hypersetup{pdftitle={\ThesisTitle}}
\hypersetup{pdfauthor={\ThesisAuthor}}
\hypersetup{pdfkeywords=\Keywords}
\hypersetup{urlcolor=blue}

% Definitions of macros (see description inside)
\include{macros}

% Title page and various mandatory informational pages
\begin{document}
\include{title}

%%% A page with automatically generated table of contents of the master thesis

\tableofcontents

%%% Each chapter is kept in a separate file
\chapter*{Introduction}
\addcontentsline{toc}{chapter}{Introduction}

Querying is the essential utility of each database. The same applies to chemical databases whose size is growing rapidly and therefore there is a pressure on efficiency of the querying process. Since the chemical compounds are typically represented as graphs, the most common queries over chemical databases are exact match query, shortest path search, similarity search and substructure search which are usually used in graph databases. The latter will be the main point of interest in this thesis.\\

The goal of this thesis is to compare the efficiency of querying methods which have been already proposed in other papers. This includes comparison of algorithms and also the utilization of native query mechanisms of both graph and relational databases.\\

We will focus on the general performance of these approaches as well as on the particular cases where some approach might be better than others.


\section*{Structure of the Thesis}
\addcontentsline{toc}{section}{Structure of the Thesis}

This thesis is divided into three main parts. In the first part we will analyze the subgraph querying problem. We will define the basic terms, list the algorithms for resolving subgraph isomorphism problem and most importantly we will analyze the related work. \\

In the second part several hypotheses will be uttered. For their verification the author’s experimental work will be used. These experiments will be described and the issues found out during the implementation will be explored.\\

The last part of the thesis will cover the results of experimental work and the comparison with results of related researches. We will comment on the findings and propose some directions in possible following research.

\chapter{Base Terms and Definitions}

\section{Definitions}

\textbf{Definition:} \textit{Graph} $G=(V,E)$ is a ordered pair where $V$ is a set of vertices and $E \subseteq V \times V $ is a set of edges .\\

\noindent \textbf{Definition:} \textit{Labeled Graph} $G=(V,E,L_{V},L_{E},f_{V},f_{E})$ is an ordered 6-tuple of set of vertices $V$, set of edges $E \subseteq V \times V $, set of vertex labels $L_{V}$,  set of edge labels $L_{E}$, function assigning the vertex labels to vertices $f_{V}: V \longrightarrow L_{V}$ and function assigning the edge labels to edges $f_{E}: E \longrightarrow L_{E}$.\\

\noindent \textbf{Definition:} Graph $G=(V,E)$ is a \textit{Subgraph} of graph $G'=(V',E')$ if and only if $V \subseteq V'$,  $E \subseteq E'$ and $((v1, v2) \in E \Longrightarrow v1, v2 \in V)$. We denote it as $G \subseteq G'$.\\

\noindent \textbf{Definition:} Graph $G=(V,E)$ is an \textit{Induced Subgraph} of graph $G'=(V',E')$ if $G \subseteq G'$ and for all edges $e=(u,v) \in E'$, ($u \in V) \& (v \in V) \Longrightarrow e \in E$.\\

\noindent \textbf{Definition:} Graphs $G=(V,E)$ and $G'=(V',E')$ are \textit{Isomorphic} to each other if there exists a bijection $I: V \longrightarrow V'$ so that $(v1,v2) \in E \Leftrightarrow (I(v1),I(v2)) \in E'$.\\

\noindent \textbf{Definition:} Graph $G$ is \textit{Subgraph Isomorphic} to graph $H$ if there exists a subgraph $H' \subseteq H$ which is isomorphic to $G$.\\

\noindent The last four definitions can be extended for the labeled graphs intuitively. 

\section{Subgraph Querying}

Due to the NP-complete nature of the subgraph isomorphism problem (is one graph subgraph isomorphic to other?), we cannot expect good results using a naive approach where we test iteratively all database records to find out whether they match the query graph or not. Usually, we need to cut down the number of these tests to the minimum.\\


Most of the techniques, described later in chapter~\ref{analysis}, are working using the following pattern:
\begin{enumerate}
	\item Based on the database statistics and approach specific heuristics, construct a database index
	\item Utilizing the index structure, build a \textbf{candidate set} of graphs for particular query
	\item Use a sub-graph isomorphism algorithm to filter out false positives from the candidate set to obtain \textbf{answer set}
\end{enumerate}


As we cannot expect significant improvement in the verification step since it is a known NP-complete problem, most of our focus in the rest of this thesis will be targeted on the first two steps, i.e. index construction and its utilization for the candidate set creation.
\chapter{Analysis of Related Work} \label{analysis}

In this chapter we summarize the work done by other authors which is related to the topic of this thesis. At first, we summarize the algorithms which have been developed for subgraph isomorphism matching and their comparison. Next we describe indices which might be used for obtaining the candidate set and algorithms which are used for their construction. The next part of this chapter focuses on approaches which utilize query mechanisms of particular relational and graph databases. In the last part we provide a summary of commercially used solutions.\\

\section{Subgraph Isomorphism Algorithms}

This section does not provide in-depth comparison of available algorithms since it is not a main topic of this thesis.\\

Almost all papers related to subgraph query methods refer two algorithms - Ullmann \cite{Ullmann} and VF2 \cite{VF2}. Those two algorithms are deeply compared in the \cite{Ehrlich2012} benchmark where VF2 outperforms Ullmann.\\

In paper \cite{Lee} there is a comparison of four algorithms derived from Ullmann's algorithm. These are VF2, QuickSI \cite{QuickSI}, GraphQL \cite{GraphQL}, GADDI \cite{GADDI} and SPath \cite{SPath}. They were compared using three real-world data sets. Although all three comparisons have a different winner, it seems that the most efficient algorithm is QuickSI in an average use-case.

\section{Index Building Methods}

In the first part of this section we briefly describe algorithms for building indices on top of chemical compound databases. These are \textit{GraphGrep} \cite{GraphGrep}\cite{GrahGrep:intro}, \textit{GIndex} \cite{GIndex}, \textit{GString} \cite{GString}, \textit{GraphGrepSX} \cite{GraphGrepSX}, \textit{GIRAS} \cite{GIRAS}, \textit{C-tree} \cite{CTree} and \textit{GDIndex} \cite{GDIndex}\\

They form just a selection from a much bigger set of applicable methods and they were picked for different reasons:

\begin{itemize}
	\item The method is mentioned in a majority of relevant articles
	\item The method uses an original algorithm or data structure
	\item The method has excellent results in benchmarks
\end{itemize}

Some of them can be used in generic graph databases, some of them are very specific to the field of chemical compounds but with some effort they might be used also for other graph databases with a specific point of interest.\\

In the following sections we will briefly introduce the basic ideas behind all the previously mentioned methods.

\subsection{GraphGrep}

Very simple and intuitive indexing technique which can be used in any graph database with labeled graphs is called \textit{GraphGrep}. The presumption is that every vertex has a defined unique ID.\\

For each graph in the database there is a constructed index represented as a hash table where the key is a hashed value of a \textit{label-path} (a concatenation of the vertex/edge labels on the path) and the value is a number of unique \textit{id-paths} (a concatenation of the vertex IDs on the path) which represent a particular \textit{label-path} in the graph. In the hash table there are all \textit{label-paths} which are present in the graph up to length \textit{l}, where \textit{l} is a parameter. This hash table is called a \textit{graph fingerprint}.\\

For example the graph in Figure~\ref{fig:graphgrep} would be represented in the index with $l=3$ as depicted in Table \ref{tab:graphgrep}. The numbers in the picture represent the vertex ID, characters next to each vertex represent its label.

\begin{figure}[h]
	\centering
	\includegraphics[width=0.25\textwidth]{../img/graphgrep.pdf}
	\caption{GraphGrep example graph}
	\label{fig:graphgrep}
\end{figure}

\begin{table}[h]
	\centering
	\renewcommand{\arraystretch}{2.5}
	\setlength{\arrayrulewidth}{0.5mm
	}
	\begin{tabular}[h!]{|p{3cm} p{3cm}|p{3cm} p{3cm}|}
		\hline
		\rowcolor{lightgray}
		Key & Value & Key & Value\\ \hline
		h(A) & 1 & h(ABC) & 2\\
		h(B) & 2 & h(ACB) & 2\\
		h(C) & 1 & h(BAC) & 2\\
		h(AB) & 2 & h(BCA) & 2\\
		h(AC) & 1 & h(BAB) & 2\\
		h(BA) & 2 & h(BCB) & 2\\
		h(BC) & 2 & h(CBA) & 2\\
		h(CA) & 1 & h(CAB) & 2\\
		h(CB) & 2 & & \\ \hline
	\end{tabular}
\caption{GraphGrep example graph fingerprint}
\label{tab:graphgrep}
\end{table}

The same process is used for the query. The query itself is also a graph and therefore the hash table can be created too. Then, in the candidate set creation part, each graph’s fingerprint is compared to the query fingerprint.\\

If any value in the query fingerprint is higher than value in the graph fingerprint for the same key or when some key from the query fingerprint is missing in the graph's fingerprint, it means that this graph can be filtered out from the candidate set because we know that the query cannot be its subgraph.

\subsection{GraphGrepSX}

GraphGrepSX is an improved version of GraphGrep. It uses the very same approach for obtaining all the indexed features - it takes all the paths up to the length $l$ from all the graphs in the database. The core of the improvement is in the data structure where the index is stored.\\

Storing all the paths for each graph in a hash table is quite ineffective. Most of the paths appear in more than one graph and we do not need to store these duplicate keys more than once.\\

This method stores the paths in the suffix tree instead. Each node in the suffix tree represents a path (which is an extension of its parent) and contains a set of pairs $(graph, count)$ where $graph$ is an ID of the database record and $count$ is the number of occurrences of the represented path in the $graph$.\\

The way how the query is processed is very similar to the GraphGrep. It mines all the paths up to length $l$ from the query graph and finds the matching nodes in the index tree. For each matched node we need to check whether the number of occurrences for each graph is equal or higher than the number of occurrences in the query graph. If so, we can add this database record into the candidate set. If some path from the query graph is not in the index, we can return an empty candidate set.

\subsection{GIndex}

This method utilizes the concepts of \textit{frequent subgraphs} and \textit{discriminative fragments}. It also comes with an innovative data structure for storing the index.\\

Since the number of all subgraphs grows exponentially with the size of the graph and therefore it would be impossible to index all of them, we need to prune the number of index records to be as compact and still as efficient as possible.\\

Because of the mentioned reasons the \textit{frequent subgraphs} and \textit{discriminative fragments} concepts have a significant role.\\

\textbf{Frequent subgraphs} are all subgraphs which are contained in at least \linebreak $ minSup $ (minimum support) graphs in the database. The survey of frequent subgraph mining can be found in \cite{frequentGraphs}. Suppose we have an index from all frequent subgraphs and for each record in the index we have a set of IDs of graphs in the database in which it occurs. If the query graph $ q $ is frequent, we have the candidate set immediately. If not, we can get the candidate set as an intersection of matched graphs sets of all frequent subgraphs of $ q $.\\

Utilization of pure set of frequent subgraphs with static $ minSup $ attribute has a couple of issues. With $ minSup $ set too low, we get an enormous set of frequent subgraphs. If the $ minSup $ is too high, the candidate set can be too large (at least $ minSup $) with larger probability of false positives.\\

That is why the described method comes with size-increasing support function. It is a non-decreasing function which takes the graph size as an argument (defined as the number of edges) and returns the $ minSup $ for given size. This results in smaller $ minSup $ for small graphs (because of the efficiency) and bigger $ minSup $ for large graphs (because of the compactness). To prevent too big subgraphs in the index, it is necessary to specify a threshold starting from which the function returns infinite.\\


An additional pruning of the index can be done. There is a very high chance that frequent subgraph $ g $ will not be enough discriminative. It means that  the candidate set of $ g $ is not significantly smaller than the intersection of candidate sets of its subgraphs.\\

\textbf{Discriminative fragments} concept brings a new metric. It measures how much discriminative the frequent subgraph is in comparison to the set of its subgraphs in the index. The discriminative ratio is defined as

$$ \gamma = \frac{|\bigcap _{i} D_{f_{\varphi i}} |}{| D_{x} |} $$

\noindent where $D_{x}$ is the set of graphs containing $ x $ and $ D_{f_{\varphi i}} $ is the set of graphs which contain subgraphs of $ x $ which are in the index. If the discriminative ratio is close to 1, we know that the discriminative power is low.\\

\textbf{gIndex} is a prefix tree data structure. Its nodes are of 2 types - \textit{discriminative} and \textit{redundant}. Each node’s key is a text string which represents the subgraph. It is serialized and canonized based on special application of DFS algorithm. This technique is called \textit{DFS Coding} and is described in \cite{gspan}.\\

Discriminative nodes are both frequent (based on given \textit{size-increasing support function}) and discriminative (based on specified $ \gamma $) and they contain a list of IDs of all graphs in the database which contain the particular subgraph. Redundant nodes are present just to satisfy the structure of the \textit{gIndex} tree.\\

The root of the tree is an empty graph, whose candidate set is the whole database. Level 1 of the tree is the set of vertices (graphs of size 0). Each node in the tree (from level 2) has 1 more edge than its parent (because of the canonization it has its parent’s key as its prefix).\\

It would be very inefficient to check all subgraphs of a query graph. But, we know that if subgraph $ g $ is not present in graph $ G $, then no superstructure of $ g $ is present in $ G $. Also, we know that if $ g $ and $ h $ are subgraphs of $ G $ and $ g \subset h $, then the candidate set generated by $ h $ is a subset of candidate set generated by $ g $ and therefore it has a bigger pruning power and usage of $ g $ is redundant.\\

From the two previously mentioned statements it is apparent what is the search algorithm. We need to enumerate all fragments of query graph $ q $ starting from 1-node fragments and iteratively enlarge the fragments by adding 1 edge each time. We stop this process at the point where the fragment is not in the index anymore.\\

Each of the fragments which were created in the last iteration can be found in the index. We only need to check whether the matched node in the index is discriminative or redundant. If it is redundant, we find the closest discriminative node on the path to root. Having the set of matched discriminative nodes in the tree, we compute an intersection of their sets of matched graphs in the database to get the desired candidate set.

\subsection{GIRAS}

As \textit{gIndex} comes with an idea of indexing frequent and discriminative fragments, GIRAS indexes rare and discriminative fragments. The idea is to get higher pruning power and put the indexing focus on the graph features which are specific for a particular record in the database. Ultimately, to have a unique index for each graph in the database. This leads to much smaller index size.\\

For getting the rare fragments it utilizes the modified version of \textit{gSpan} algorithm \cite{gspan}. Although, the original \textit{gSpan} is designed to get all subgraphs whose support in the database is \textit{n or higher}, the modified version finds all the subgraphs whose support is equal to \textit{n}.\\

The modified \textit{gSpan} utilizes minimal DFS codes which were already described in \textit{gIndex} section. It starts with an empty DFS code and in each call it finds all the possible right-most extensions from the whole database. For all of them it finds out whether they are minimal DFS codes and, if so, it checks what the support of this subgraph is. If it is equal to the specified support $f$, the subgraph is added into the result set. If the support is higher, we continue recursively.\\

Note that it returns only the minimal rare substructures with a given frequency. This is important since the extensions of these minimal rare substructures with the same frequency would not give us any more pruning power but it would increase the index size significantly.\\

The GIRAS itself then calls the modified gSpan. It starts for $f=1$. After each call of modified gSpan it checks which database records are represented by the result set of gSpan. If there are database records which are not indexed yet, the modified gSpan is called iteratively with $f+1$. Once there are all database records indexed, we are finished. The last $f$ is called $f_{min}$ and it is the threshold defining the meaning of rare substructure.\\

Although it is not discussed in the paper \cite{GIRAS} what data structure it uses for the index representation, we found out from the source code obtained from Dr. Azaouzi, the author of the described research, that it uses very similar data structure which was described in \textit{gIndex} section, as well as the same technique for the querying process.

\subsection{GString} \label{gstring}

All other methods can be used in any graph database. On the other hand, GString method is very specific for the organic chemical databases (but can be internally modified to support different graph databases with specific content).\\

The main ideas come from the knowledge of common structures of the graphs in the database. The chemical compounds consist of 3 types of semantic structures - paths, cycles and stars (a central node with a fan-out). Each chemical compound can be converted into a graph whose nodes are not atoms but one of the mentioned structures. This converted graph is significantly smaller than the original one.\\

The other observation is that we can omit the hydrogens since their number can be easily computed and we can omit the labels of carbon atoms and single (saturated) bonds.\\

Based on previous preliminaries, each graph in the database can be shrinked to the graph of common structures. Each node contains 3 types of information:

\begin{itemize}
	\item \textbf{Type} - path, cycle or star
	\item \textbf{Size} - For path and cycle it is the number of nodes, for star it is the fan-out
	
	\item \textbf{Triple} $ <n_{n}, n_{b}, n_{e}> $ where:
	\begin{itemize}
		\item $ n_{n} $ is the number of non-carbon atoms
		\item $ n_{b} $ is the number of branches (connected paths of the length 1)
		\item $ n_{e} $ is the number of double or triple bonds
	\end{itemize}
	
\end{itemize}

For each such graph we can get a set of all paths up to length $ l $. The index structure of GString method is a suffix tree of these paths, where each node is identified by tuple $ <Type, Size> $ and contains a set of pointers to the \textit{detail table} where quadruples $ <n_{n}, n_{b}, n_{e}, id> $ are stored for matched nodes from a particular graph. The suffix tree is built from all paths up to length $ l $ from all graphs in the database.\\

The candidate set is obtained as follows. The query graph itself is translated to the common structure graph by the same process which was utilized for index building. Then we just identify suffix tree nodes which were visited and use the pointers to the \textit{detail table} in such nodes. The graph is added into the candidate set if it is represented in each visited suffix tree node and if the triple $ <n_{n}, n_{b}, n_{e}> $ satisfies the query.\\

It means that for cycles, the $ n_{n} $ and $ n_{e} $ has to be equivalent in both query and database record, $n_{b}$ has to be equal or lower in the query comparing to the database record. For the paths and stars all three attributes has to be same or lower in the query.\\

Note that the answer set of this method can be different from previous methods. Let us take a path of four carbons $ c-c-c-c $ as an example of a query and assume that the benzene (cycle of six carbons) is a part of the database. The previous methods marks the benzene as a \textit{match}. On the other hand the GString will filter it out from the candidate set because it finds out that its \textit{common structure} graph is completely different.\\

However, this is a correct behavior for the chemical compound database since we can expect that if somebody asks for a path of four carbons, he or she does not expect a benzene as a result since cycles and paths have different semantics.

\subsection{C-Tree}

Contrary to the previous methods, this one does not utilize the fragments of the graph to find the candidate set. It builds the state-of-the-art tree structure where the nodes are \textit{closures} of their children so they contain the same substructures as their whole subtrees. Also it comes with the term of \textit{pseudo sub-isomorphism} which is similar (and weaker) to subgraph isomorphism but it can be verified in polynomial time.\\

The core of the C-tree method are the graph closures. Let $ G $, $ G' $ be graphs and $ m $ be the mapping between them (graphs can contain dummy nodes for enabling mapping between graphs of different size). Let $ v $, $ v' $ be nodes from $ G $ or $ G' $, respectively and let $ m(v) = v' $. Vertex closure which corresponds to $ v $ and $ v' $ then contains a union of labels of $ v $ and $ v' $. The very same approach is used for edges.  Let $ e $, $ e' $ be edges from $ G $ or $ G' $, respectively and let $ m(e) = e' $. Edge closure which corresponds to $ e $ and $ e' $ contains a union of labels of $ e $ and $ e' $. \textbf{Graph closure} of graphs $ G $ and $ G' $ is a tuple $ (VC, EC) $ where $ VC $ is a set of vertex closures and $ EC $ is a set of edge closures. Note that $ G $ and $ G' $ can be both graphs and graph closures.\\

Several approaches how to get the mapping $m$ are described in \cite{CTree} and we will not describe these in the this section to not dive too deep into the technical details.\\

The \textbf{C-tree} data structure is a tree where leaf nodes are graphs from the database and every internal node is a graph closure of its children. Each node has at least $ m $ children unless it is root, $ m \geq 2 $, and each node has at most $ M $ children, $ \frac{M+1}{2} \geq m $. All operations with the tree are done in polynomial time and their implementation is analogous to those on R-trees \cite{RTrees}\\

The idea of the method is to approximate the subgraph isomorphism by a weaker statement, \textbf{pseudo subgraph isomorphism}, which can be tested in polynomial time. An important note is that pseudo subgraph isomorphism can be tested on both graphs and graph closures.\\

Full description of the theory behind the pseudo subgraph isomorphism would be too exhaustive for the purposes of this thesis. Very briefly, the idea is to construct a bipartite graph $G$ between vertices of graph $G_{1}=(V_{1}, E_{1})$ and vertices of $G_{2}=(V_{2}, E_{2})$. There is an edge between $v \in V1$ and $u \in V2$ if \textit{breadth-first search tree} around $v$ with the paths up to the specified length $n$ is isomorphic to the one around $u$. If $G$ has a semi-perfect matching, $G_{1}$ is \textit{level-n pseudo subgraph isomorphic} to $G_{2}$\\

The authors of C-tree are also proposing a recursive algorithm which can effectively obtain the information whether two nodes should be connected by an edge in the previously mentioned bipartite graph for the level $n$ based on the bipartite graph for the level $n-1$.\\

The candidate set creation process utilizes the C-tree. It goes from the root to leafs and every time it finds out that a query is not pseudo subgraph isomorphic to some node, this node and its subtrees are pruned out. Leafs which are pseudo subgraph isomorphic to the query are added to the candidate set.\\

The main advantage of this method is that contrary to the previous methods, this one does not loose information during the index creation time. It does not count with paths or any other fragments, the closure tree does contain all the information about all the graphs in the database. This helps to increase the level of the pruning during candidate set creation.

\subsection{GDIndex}

This method's approach is quite different to the previous ones. It tries to completely omit the verification step and therefore computationally hard usage of any subgraph isomorphism detection algorithm. It is achieved by all the subgraphs of all database records.\\


It uses two structures in the index:

\begin{enumerate}
	\item Directed acyclic graph (DAG) of all subgraphs. Each node in the DAG represents a specific connected subgraph. Each such node contains also the information whether it refers an actual record in the database. There is a directed edge from node $ N $ to node $ M $ if $ N $ is a subgraph of $ M $, $ N $ contains exactly 1 vertex less than $ M $ and $ N $ is an induced subgraph of $ M $.
	
	\item Lookup hash table of subgraphs. There is a record in the hash table for each node in the DAG. For hashing, the canonical form of the graph is defined. This canonical form is derived from the adjacency matrix.
\end{enumerate}

Both index building and querying is straightforward. To build the index we just take each graph, add it to the DAG and by gradual removing of its vertices we repeat the same procedure for all its subgraphs. In each step we just need to check whether such node already exists in the DAG which we can easily achieve using the lookup table.\\

To reduce the number of subgraphs, the canonization technique is introduced and from all isomorphic subgraphs only one is used in the index. This canonization technique is very similar to the DFS codes described in \textit{gIndex}, however, instead of minimal DFS code it is using maximal adjacency matrix serialization (but both approaches are equally strong and have the same computational difficulty).\\

Querying is even simpler. All we need to do is to create a canonical representation of the query graph and use the lookup table. If the particular record is not present in the index, we know that the candidate set is empty. If there is such node, we recursively iterate through all its descendants in the DAG and find all pointers to the database graphs. Since we are using hash table, we can get false positives. Therefore, for each record in the matched row of a hash table we need to compare the exact canonical code and we will use only the the record which is exact match.\\

The big advantage of this method is that we do not have to do the NP-complete subgraph isomorphism test since we store the subgraphs in the index and we have the canonical representation.\\

What we have found as a missing piece (and there is no information about this case in the paper) is that the query does not have to be an induced subgraph of any node in the database. It can be more sparse. In this case we cannot expect the exact match of the canonical code and therefore we cannot expect any results.\\

The possible solution to fix this problem would be to index all the subgraphs instead of just induced ones. On the other hand that would have serious impact on the index size.

\subsection{Benchmark Results}

\textit{GraphGrep}, \textit{GIndex}, \textit{GString} and \textit{C-Tree} have been compared in \cite{GString}. As the testing data set the AIDS Antiviral Screen Dataset \cite{AIDS} was used. It contains 43 000 molecules with an average number of 25 vertices.\\

All measured metrics except for the speed of index creation had the same winner. The \textit{GString} algorithm outperforms the others in the size of index, accuracy of the candidate data set and the search time.\\

On the other hand, in \cite{GraphGrepSX} we can find the benchmark of the \textit{GraphGrepSX} method which looks like a more generic version of \textit{GString}. While in \cite{GString} \textit{GString} outperforms \textit{CTree} just by few percents, in \cite{GraphGrepSX}  \textit{GraphGrepSX} outperforms the \textit{CTree} by the two levels of magnitude despite larger candidate sets.\\

In \cite{GDIndex} there is a comparison of \textit{GDIndex} and \textit{C-tree} where \textit{GDIndex} significantly outperforms \textit{C-tree} in all measured metrics - the size of index and its construction time and the search time.\\

What we may question is that how \textit{GDIndex} would perform over a database with larger graphs such as the AIDS dataset which was used in experimental parts of all other methods.\\

In \cite{GIRAS} we can find a benchmark of the \textit{GIRAS}, \textit{C-tree}, \textit{gIndex} and couple of other approaches. On the AIDS dataset \textit{GIRAS} outperforms \textit{gIndex} and \textit{C-tree} in all query sizes. In the dataset with bigger graphs, \textit{GIRAS} outperforms the other two methods only in larger query sizes (12 vertices and more).\\

What is not measured in \cite{GIRAS} is the size of index and time needed for index construction.

\section{Database Management Systems Utilization for Subgraph Querying}

Surprisingly we have not found many articles about substructure querying in DBMS using just their native way how to structure data and their specific query language.\\

The first approach \cite{SQL} we found is about the utilization of relational database management system and SQL queries. The second one \cite{Hoksza} is referring about utilizing a graph DBMS, Neo4j \cite{Neo4J}, and its query language Cypher.

\subsection{SQL Substructure Search}

Contrary to typical subgraph matching algorithms which use variations of the depth-first-search algorithm, the authors of \cite{SQL} come with an SQL based solution which utilizes the principles of the BFS.\\

In the database the molecules are described as follows. The database contains 3 tables - molecules, atoms and bonds. The bonds have an extended type column which is a string identifier that identifies bond type and types of both end atoms type (e.g. there is a unique identificator of two carbons connected by double bond).\\

The bond table has three indices built on top of it. The first one is built for bond type which helps us to do efficient filtering, the second one is built for \textit{atom1\_id} column (a reference to the atoms table) which helps us to get all neighbours for each atom. The last index is built based on unique identifier of records in bond table by atom pairs.\\

When the substructure query is obtained, the minimal spanning tree is constructed. The value of each edge depends on the statistics of the database. We can say that the most rare atom-bond-atom edge has the lowest value. Also in this tree we find a root node which has the least valuable edges on it. This spanning tree will help us to construct an efficient SQL query, because thanks to the spanning tree minimality and the root selection the constraints (edges) with the highest probability of failure will be checked first.\\

The query itself uses only the edge table. It starts from the root of the spanning tree. For each edge there is a specification of an extended bond type and specification of a join to other instance of edge table. At the end there are edges which are not a part of a spanning tree.\\

As an example we can use a subgraph query where we want to find all structures which contain $ O=C-N $. The bond $ C-N $ is more rare in the sample database and therefore this bond is described as the first one in the query. The query itself would look as follows:


\begin{verbatim}
SELECT b1.compound_id, b1.atom1_id, b1.atom2_id, b2.atom2_id
FROM bonds b1, bonds b2
WHERE b1.bond_type = "C-N" and
      b2.atom1_id = b1.atom1_id and 
      b2.bond_type = "O=C"
\end{verbatim}

\noindent where $ C-N $ means carbon and nitrogen connected by a single bond and $ O=C $ means oxygen and carbon connected by a double bond.\\

This example is quite simple. On the other, hand we need to build an SQL query which describes the whole \textit{Constrain Satisfaction Problem}. It means that for each pair of bonds, we have to define whether their atoms do or do not have the same IDs.\\

Where it is possible, we can force usage of built indices. For the first edge we should use the index built for the bond type column. For other spanning tree edges we should use the index for $ atom1\_id $ column which literally does the BFS. For edges outside the spanning tree we should use the index built for $ atom1\_id $, $ atom2\_id $ pair since we already know the IDs of both atoms of the edge we need to check.

\subsection{Neo4j Substructure Search}

Hoksza et al. in \cite{Hoksza} describe their case-study of mining the protein graphs. They use the Neo4j graph DBMS to store the protein database and query it by the Cypher language.\\

They found out that the query time is factorial with respect to the number of edges in the query. Beginning from size 15, the queries were impossible to execute in a reasonable time and therefore they recommend the usage of Neo4j only for small subgraph queries.\\

They have also tried to compare their results with results for an SQL database. However, the SQL results significantly outperform Neo4j. But the comparison is not fair enough since the SQL approach used pre-computed neighborhood relations and therefore had a significant advantage in comparison with Neo4j.\\

However, based on this paper we can be pessimistic in case of Noe4j utilization, we should keep in mind that the database had a different structure comparing to our molecule databases which are the target of this thesis. Graphs used in the experiment have an average size of more than 500 edges. On the other hand, typical molecule databases contain significantly smaller graphs and therefore we cannot be sure that the numbers from the mentioned paper can be applied also for such databases.

\section{Commercially Used Solutions}

In this section we introduce three real-world solutions. The first one is the AMBIT project \cite{Ambit} which offers chemoinformatics functionality via REST web services. One of the functionality is, of course, the substructure search. This project represents a standalone solution - the querying is not dependent on any particular database management system.\\

The second solution, JChem Cartridge \cite{JChem}, is an example of an Oracle cartridge \cite{cartridge}. The reason why we picked this cartridge from the set of existing ones is that it has the best results in the benchmark presentation at \cite{benchmarkPresentation}.\\

The third solution, ABCD Cartridge \cite{ABCD}, is a pure commercial one developed by the Johnson \& Johnson company \cite{JJ}. We picked this one because its architecture is well described in \cite{ABCD} despite the software is not publicly available .

\subsection{AMBIT-SMARTS}

AMBIT-SMARTS is a Java based software built on top of the Chemistry Development Kit (CDK \cite{CDK}). It implements the whole SMARTS querying language specification \cite{SMARTS} for querying chemical databases. It uses two indices. Both are in the form of a bitstring which is stored for each record in the database.\\

Each bit in the first bitstring represents whether some structure is a part of the particular record. The structures are of two kinds.\\

The first set of structures is selected automatically based on the database content. It considers each atom’s topological layers. The first topological layer is the atom and all its neighbours. n-th topological layer is the whole (n-1)-th layer and some or all of its neighbours. All such structures up to a selected layer level are recorded. Structures which are a part of at least ~50\% of database records are considered as those which will be represented in the bitstring.\\

The second set of the structures represented in the first index is selected by the database administrator who should be aware of what types of queries are most likely to be used in such database.\\

The second bitstring represents all paths up to length 7. Because the number of these paths is enormous, they are not represented directly in the bitstring, but they are at first hashed and this hashed value is added (by logical OR) to the bitstring. This concept is called \textit{fingerprints} and it is described in \cite{fingerprints}.

\subsection{JChem Cartridge}

The JChem Cartridge is a part of the JChem package from ChemAxon \cite{Chemaxon}. It allows users to build their chemical database in the Oracle database easily. A part of the cartridge contains tools for chemical formats conversion, similarity search and sub-structure search. It also implements functions for SMARTS queries.\\

With regards to the substructure search it filters the database based on the fingerprints which are present for every molecule. It uses the hashed fingerprints similarly to the AMBIT-SMARTS. The keys for hashing are:

\begin{itemize}
	\item All paths in the molecule up to a specified length
	\item The branching points (atoms with degree higher than two)
	\item All cycles
\end{itemize}

\noindent The fingerprint itself is generated based on 3 user-defined parameters:

\begin{itemize}
	\item The length of the fingerprint
	\item The maximum path length (how long paths are used for generating the hash keys)
	\item How many bits are set to 1 for each hash key
\end{itemize}

In the documentation there is stated that for the substructure search the optimal values in most cases should be 512 bits long fingerprints, the maximum path length set to 5 or 6 and the number of bits per hash key set to 2.\\

The cartridge also has a tool for analyzing the efficiency of the fingerprints. As a good metric the idea of \textit{darkness} is used. Darkness is defined as a ratio between numbers 0 and 1 in the fingerprint. The analysis tool provides the user with information about the lowest, average and highest darkness in the database and also provides a distribution. The darkness should be as low as possible, highest values should not exceed 80\%, but best performance is expected under 66\%.

\subsection{ABCD Cartridge}

ABCD is an integrated drug discovery informatics platform developed by the Johnson \& Johnson Pharmaceutical Research \& Development, L.L.C. It consists of a set of algorithms for subgraph isomorphism checking and index building and an interoperability layer, cartridge, for the Oracle database which enables the RDMS to use the algorithms and indices during the SQL query evaluation.\\

For the filtering it uses a set of hashed fingerprints. There are 5 types of fingerprints which are used for each molecule - atom, edge, ring, path and cluster fingerprint. For each type there is a different algorithm which generates the hash keys. Also for each hash key, the number of occurrences of a particular feature is stored.\\

Contrary to AMBIT it does not store the fingerprints for each record in the database. It utilizes the concept of inverted bitstrings.\\

The algorithm proceeds as follows. Every molecule in the database is analyzed and the set of hash keys along with the number of occurrences in that molecule are computed. The information for each key is stored as a triplet $ {h,c,m} $, where $ h $ is the hash code, $ c $ is the number of occurrences, and $ m $ is the ID of the molecule in the database. The list is then traversed and for each unique hash code, $ h $, a series of binary masks, $ M(h,cmin) $, are defined, where $ M(h,cmin) $ contains the IDs of the molecules for which the hash code $ h $ occurs at least $ cmin $ times.\\

For more compact representation of the inverted bitstring there are three types of their representation where $ N $ is the size of database and $ K $ is the number of database records in the matching set:

\begin{itemize}
	\item If $ K < \frac{N}{32} $ then the representation is an array of IDs of database records which belong to the set.
	\item If $ (N - K) < \frac{N}{32} $ then the representation is an array of IDs of database records which do not belong to the set.
	\item Otherwise it is stored as a classing bitstring where n-th bit represents whether n-th record belongs to the set.
	
\end{itemize}



















\chapter{Experimental Work}
\section{Introduction}
During the research of the related work, many questions arise. The papers are usually very brief and they miss a lot of implementation details. Sadly, even if we tried to contact the authors, we did not get the original source code for the described methods nor for the described benchmarks. The only exception is the \textit{GIRAS} method where we were successful in contacting its author and we do have the complete implementation.\\

All the benchmarks we mentioned in the previous chapter were a part of the papers which describe each particular method. Knowing that we cannot be much surprised that the each presented method outperformed the others. The question is whether we do get the same results on different data sets.\\

The other interesting question is how the winners of the various benchmarks would perform on the same data set. For example, when \textit{GString} outperforms the \textit{C-tree} just by few percents in \cite{GString} and \textit{GraphGrepSX} outperforms \textit{C-tree} by two levels of magnitude, we cannot implicitly say that \textit{GraphGrepSX} would outperform \textit{GString}. There might be three reasons why this presumption might be wrong:

\begin{itemize}
	\item The lack of knowledge of the tested data set. In most the papers there is an information which dataset has been used. On the other hand, there is usually no information about which part of the dataset has been used since the dataset is usually cut down to only a small part of the original size. Moreover, not all the benchmarks are using the same datasets at all.
	
	\item The lack of knowledge about the implementation of the verification step. In non of the mentioned papers is an information about which algorithm has been used for the final subgraph isomorphism testing. This can cause quite a significant difference in the final query measurements (although it cannot influence in the candidate set time computing).
	
	\item We do not even know how much time the authors spent on the optimization of the code itself. Whether they cared more about the code readability and maintainability of the code or whether they did try to optimize the code as much as possible. Moreover, we do not know anything about which languages and compilers have been used.
\end{itemize}

What we did not find at all is some comparison of the performance of the described indexing techniques and utilization of SQL or NOSQL databases. It might be interesting see how significant difference in performance we get when we use very graph specific technique comparing to the very generic ones which the databases offers.\\

In the following sections we will describe what hypotheses do we found interesting to prove or disprove and we describe the process and the implementation of those proves.\\

What is probably fair to mention is that due to the brevity of the related work we cannot be sure whether we did not omit some important part of the algorithms. There has been a lot a situations where we had to improvise since we found out that some very important implementation detail has been omitted in the method descriptions. These cases will be described in following sections as well. Although, we did implement all the methods with opened mind without any endeavor to make some method better or worse, we cannot guarantee that we did not do any mistake or bad implementation decision which can influence the final benchmark results.

\section{Hypotheses to be verified by the experimental work}

In this section we will list several hypotheses which came to our mind during the related work research.

\subsection{Hypothesis 1: GString vs GraphGrepSX}

\textit{GString} and \textit{GraphGrepSX} are using very similar data structures for indexing the database. The main difference is that \textit{GraphGrepSX} is using all graph paths, \textit{GString} is using all paths in the condensed graph. Also \textit{GString} is using heuristics which are very specific for the our field of research, i.e. the organic chemical databases.\\

That being said, we would expect that the index size of \textit{GString} will be significantly smaller due to the condensed graph usage. Also we would expect that due to the specificity of \textit{GString}, it should outperform \textit{GraphGrepSX} which can be used for any graph dataset.

\subsection{Hypothesis 2: GIRAS performance for large queries}

For small queries (of size 4 and 8) the performance of \textit{GIRAS} is about the same as \textit{C-tree}. On the other hand, for larger queries, the performance is ten times better comparing to \textit{C-tree} and even better results are there for the candidate set sizes. What we may question is how it will perform comparing to \textit{GString} and \textit{GraphGrepSX}. From the benchmarks comparisons we may say that \textit{GraphGrepSX} would be the winner. On the other hand the candidate set size should be much better for \textit{GIRAS}. Our guess is that despite the better candidate set size \textit{GIRAS} will not perform better than \textit{GraphGrepSX}.\\

Also, what might be interesting to measure is the index building time for \textit{GIRAS} since it is not mentioned in the paper and the algorithm seems quite computationally complicated.

\subsection{Hypothesis 3: How the SQL and graph oriented databases perform in comparison with the domain specific solutions}

We may question what performance we may get when we use some SQL or graph database. In this case we do not need to implement any special algorithms for index building, we just use the possibilities of the databases, i.e. create a query which describes the subgraph and in case of SQL databases to build the indices to help the query process.\\

We expect that the domain specific indexes will perform much better. But it might be very interesting to see how significant is the performance gap. Also, we may expect that graph databases will perform better compared to the SQL databases since they are usually optimized for storing and querying graphs.

\section{Description of the Experimental Work}

In this section we will describe the implementation details of the experimental work. Based on the uttered hypotheses we have implemented:

\begin{itemize}
	\item \textit{GraphGrepSX} and \textit{GString} algorithms
	
	\item Adapter for the \textit{GIRAS} implementation obtained from Dr. Azaouzi to be working on the same dataset
	
	\item Tools for inserting and querying the SQL and graph database
\end{itemize}

All the implementation has been written in Java language\cite{java}. Most of the work is using Java version 10, graph database adapter is using Java version 8 due to the technology dependencies.\\

For the chemical database parsing we are using Chemistry Development \break Kit\cite{CDK} version 2.1.1, a Java library for working with chemical formats and data structures.\\

In case of verification step for the \textit{GraphGrepSX} and \textit{GString} algorithms, we are using the \textit{SMARTSQueryTool} from Chemistry Development Kit. It uses the \textit{Ullmann}\cite{Ullmann} algorithm inside.

\subsection{GraphGrepSX} \label{graphgrep-implementation}

Since the \textit{GraphGrepSX} algorithm is very simple, the implementation was quite straight-forward.\\

We had to do only one change in the algorithm to make it applicable to our use-case. The original description of the algorithm expects that the suffix tree represents the vertex label paths. Since we need to represent even the edge labels we have changed the original suffix tree presumption so that the odd levels of the suffix tree represent the vertices and the even levels of the suffix tree represent the edges.\\

The previous statement does not affect the {maximum path length parameter $l$ of \textit{GraphGrepSX} algorithm. It is still valid that this parameter sets the maximum length of the index path with regards to the number of vertices, therefore the index tree will have depth up to $2l - 1$.

\subsection{GString}

In contrary to the previous paragraphs, the \textit{GString} algorithm description offers a wide range of pieces which were not described at all. The most of the unknown parts are related to the original graph reduction process where the graph representing the atoms and bonds is transformed into a graph consists only from nodes representing cycles, stars and paths and edges represent just the connection between these structures.\\

The first issue which we faced was the process of extracting the cycles from the original graph. In the algorithm description there are no references on how to extract the cycles nor which method should be used. The obvious issue is that the cycles are not necessarily independent. They can share both vertices and edges and in some cases the vertices and edges can be shared even by several cycles.\\

After some research we have found out that the Chemistry Development Kit has an utility for retrieving \textit{MCB - Minimum Cycle Basis} (also known as \textit{SSSR - Smallest Set of Smallest Rings}) described in \cite{Bauer}. \textit{Cycle Basis} is defined as a set of cycles by which you can express any other cycle present in particular graph as an result a symmetric difference operation on the basis.\\

The \textit{MCB} is defined as a cycle basis which consists of the shortest possible cycles. A good example might be naphthalene which you can see on Figure~\ref{fig:naphthalene}. It contains three cycles, two of size 6 and one of size 10, and any couple of these can serve as a cycle basis. On the other hand, there is only one \textit{MCB} which consist of two cycles of the size 6.\\

On this picture it is also clearly visible why we cannot use all the cycles. If all three cycles would be represented in \textit{GString} graph, it would be very unclear what is the the relationship between these cycles and how they should be connected in \textit{GString} graph. Also, it may lead to false positives from chemistry point of view because naphthalene consist of two aromatic cycles and it does not make sense to include an information about the cycle of the size 10.

\begin{figure}[h]
	\centering
	\includegraphics[width=0.25\textwidth]{../img/naphthalene01.png}
	\caption{Molecule of Naphthalene}
	\label{fig:naphthalene}
\end{figure}

The \textit{MCB} finder utility requires specification of the maximum cycle size parameter. This parameter defines a threshold above which the cycles are not considered as cycles. When we tried to set this threshold high enough to not omit any cycle in the testing database, we had big issues with performance and in some cases the process dies on the lack of memory. Since the target of this thesis is to measure the performance in usual use-cases, we have decided to set the threshold to 10 which should cover the vast majority of real cases.\\

Bigger cycles are described as paths of the length equal to the cycle size. These paths begin at each point in which the cycle interfere with another \textit{GString} structure. For each such interference there is are two paths, one in each direction\\

Another question which arose is how to set the threshold which defines the minimum degree of an atom to be considered as star. The original thought was to set the threshold to 3. The reason was that if we set this threshold to higher number we get another problem to solve - how to handle path joints. We can demonstrate this problem on methyl propionate on Figure~\ref{fig:methyl-propionate}\\

\begin{figure}[h]
	\centering
	\includegraphics[width=0.25\textwidth]{../img/methyl-propionate.jpg}
	\caption{Molecule of Methyl Propionate}
	\label{fig:methyl-propionate}
\end{figure}

We can see that there are six possible paths, two are between the carbons (one from each side), and four between each carbon and oxygen (again, one from each side). If we define the threshold of stars to three, we do not have to handle such situations because in our algorithm, we extract the stars first and then we are finding paths connected to already found stars and cycles. In this case we would have 1 star (the atom in the middle) and two paths connected to it (the oxygen connected by double bond would be considered as a branch which is by definition a path of length 1). This would simplify the algorithm quite a lot because we would not need to handle paths which are connected to another paths.\\

During the testing we found out that it is possible to use this threshold but in practice, we would lost majority of results. The reason is the same as we described in the first chapter. If we try to query a path which may be described as $C-O-C-C-C$ there is obviously such path in methyl propionate but our algorithm would filter this candidate out because it does not contain a path of size 5 but a star and two paths of length 2. Since this is a very common use-case we had to use higher threshold and develop some logic for handling the connected paths.\\

What we did was to implement DFS which finds all the paths and all of these paths are included into the \textit{GString} graph. In case of methyl propionate there would be 2 independent paths, since we do not have a connection to any other structure, we start DFS in random atom with degree 1. This is quite a special case because the molecule consist only of paths. If the methyl propionate would be connected on one end to star or cycle, there would be 2 nodes in the \textit{GString} graph - one cycle/star and two paths connected to this structure.\\

The rest of the algorithm mimics the \textit{GraphGrepSX} implementation including the notes described in \ref{graphgrep-implementation}.

\subsection{SQL Database}

We have based our implementation on the proposal in \cite{SQL}. We have chosen the Oracle Database 12c. For the Java API we have used Oracle Database JDBC driver 12.2.0.1.\\

Based on the mentioned paper we have designed our table with 5 columns - \textbf{ATOM1\_ID}, \textbf{ATOM2\_ID}, \textbf{BOND\_ID}, \textbf{BOND\_TYPE} and \break \textbf{COMPOUND\_ID}.\\


The implementation itself is quite straightforward and it consist of two parts. The first part is a routine for the database creation. In this routine we just iterate through the whole database and for each molecule, at first, we iterate through all its atoms and assign an unique ID to each of them, later, we iterate through all the bonds and for each we create one \textbf{INSERT} statement.\\

We already have the IDs of atoms and compound ID (this comes from the original chemical DB on the input), we generate an unique ID for bond itself. Type of bond consists of the type of each atom at the bond's end and the type of the bond itself. Each bond, if it is not symmetrical, is represented by two rows in the database because we need to make the graph representation undirected. During the process of inserting the bonds we are updating the in-memory statistics - we are maintaining the count of rows for each bond type.\\

The inserts are happening in batches. We did test the performance and found out that batch size of 50 rows for one \textbf{INSERT} statement is quite optimal.\\

The second part of the implementation is the query building. As proposed in \cite{SQL}, at first we build the minimal spanning tree of the query graph. The edge value is based on the database statistics which we gather during the insert phase. For spanning tree construction we have implemented Kruskal algorithm.\\

Then, in the spanning tree we find the edge with lowest value and from this edge we start a BFS algorithm and for each edge we add the rule into the \textbf{SELECT} statement. We also need to mark all the neighbours by stating that atom ID of one edge is equal to the atom ID of the neighbour edge. The same we have to do for non-neighbours. For each such pair we have to explicitly state that their atom IDs are not equal. The same we have to do for the bonds, we need all the bonds unique so we have to state for each pair of bonds that their IDs are not equal.\\

Since we are interested only in the information whether the subgraph is present in particular compound, we start our \textbf{SELECT} statement with \textbf{SELECT DISTINCT b0.COMPOUND\_ID FROM ...} which returns the set of compounds matching the subgraph query which is exactly the result we need.

\subsection{Graph Database}

As observations of paper \cite{Hoksza} state that the Neo4j is not performing well in subgraph querying, we have tried to look for alternative graph databases which can perform better. We found out graph analytic tool \textit{PGX} \cite{pgx}.\\

It is not a graph database in its core meaning. It is a toolkit for graph analysis with an ability to load and store graphs from / to various data formats. It does support a SQL-like query language called \textit{PGQL} \cite{pgql} and it advertise a scalable solution with a focus to high performance.\\

Since it is a analytic tool and not a real database, it does not support ACID transaction model as other databases do but for the purposes of this thesis it does not take serious role.\\

Oracle did a benchmark which compares the performance of subgraph matching in PGX and Neo4j. The results are available at presentation \cite{pgx-neo4j} on slide 31. Note that the benchmark is comparing the results on so-called \textit{hot data}. In other words, it makes sure that the graphs which are being queried are already loaded in memory to prevent result inconsistencies due to the data fetching from the disk.\\

The results of this benchmark are quite convincing. Although there are huge differences of result times according to the query size, PGX outperforms Neo4J in all categories.\\

The first issue we had to solve was that although Oracle is offering windows batch files for starting PGX, we were not successful to successfully run the database. To make the results of the performance measuring as precise as possible, we did not want to use other hardware or different operating system for executing the performance testing of PGX than the hardware and system used for other performance testing of other methods.\\

As a viable compromise we have decided to use the Windows 10 Subsystem for Linux utility \cite{wsl} and we have downloaded the Ubuntu system to be used in this manner. On Ubuntu there were no issues with the PGX database usage.\\

The client side has been implemented in Windows environment using PGX Java client library. The implementation is quite straight-forward. Instead of creating the graph in PGX for every single molecule in the database, we create one huge graph where each graph component represents one molecule. This helps the performance since we do not have to load and store the a huge number of small files, we are executing query on just one graph instead.\\

For each vertex we generate a unique ID, use label to mark the representing atom symbol and we store a molecule ID as a vertex's property. For each edge we use label to mark the type of represented bond.\\

For the querying we are using the PGQL language which is supported by PGX. For each atom in query graph we generate an unique ID and then for each bond in query graph we insert a rule into the query where we define that the two atoms with particular IDs and of particular type are connected by a bond of a particular type. As the last thing in the query we need to state for each pair of atom IDs that they represent different vertex.\\

As an demonstration of simplicity of PGQL usage in our case might be a query representing the path of three carbons. We start a query with the following statement which tells us that we want to select all the molecule IDs (which are being stored for each vertex) except the duplicates. Note that the $a1$ is an ID of the first atom in query graph.

\begin{center}
	\textbf{SELECT DISTINCT a1.moleculeId}
\end{center}

The following part of PGQL query describes the bonds in the query graph. The colon sign means that we are describing the label of the vertex or edge. We can see that in this case we are looking for two pairs of carbons connected by a single bond (represented by \textbf{S} character), where the \textbf{a2} atom is shared in these two bonds.

\begin{center}
	\textbf{MATCH (a1:C)-[:S]-(a2:C), (a2:C)-[:S]-(a3:C)}
\end{center}

As the last part we need to state that all atoms \textbf{a1}, \textbf{a2} and \textbf{a3} are different. Otherwise the query would match for every pair of connected carbons since \textbf{a1} and \textbf{a3} could represent the same atom

\begin{center}
	\textbf{WHERE a1 $<>$ a2 AND a1 $<>$ a3 AND a2 $<>$ a3}
\end{center}

We must admit that the work with PGQL is quite intuitive and comparing to building the SQL query it is way more user friendly. On the other hand, this is quite an expected result since PGQL is designed to query graphs and SQL is designed to query generic data.

\subsection{GIRAS}

As we were successful with the request of the original implementation of \textit{GIRAS} algorithm, there was not much implementation needed on the side of this thesis. For the measurement we are using the original solution. We had to implement only an adapter which translates the chemical database which we are using for other methods to the format - vertex and edge lists - accepted by the \textit{GIRAS} code.\\

However, during the testing we have found out that the results are not matching the results of other methods. After some investigation, we have realized that the problem is not in the implementation but in the algorithm itself. The core of the problem is in the way how the structures which are being indexed are chosen.\\

As we mentioned in the analysis chapter, \textit{GIRAS} method is trying to find the rare substructures with the condition that every graph in the database has to be represented by at least one rare subgraph. We have found out that when we create a query which should have only several results, everything works fine. On the other hand when we build a query which should match nearly whole database, we do not get any results at all.\\

We did an explicit test which proves that the algorithm cannot work properly in all cases. We have created a database with 4 molecules where each of them contain a path of four aromatic carbons but in each of them there is a unique substructure which do not contain this particular path.\\

When we have executed a query of the mentioned path we did not get any results. When we have added new molecule into the database which represents the query itself, i.e. path of four aromatic carbons, and we ran the query again, the result was that the query matches all the graphs in the database.\\

This observation invalidates the statement in \cite{GIRAS} that the indexing is complete. We may then question how the results of the performance measurements are valid. If we know that the indexing is incomplete, it should be also faster since the index is smaller and therefore it should take less time to use it. So even for the queries which results are valid, we may question how seriously we can take the performance numbers.
\chapter{Experimental Results}

In this chapter, we present the results of the experimental work. We have measured the following statistics:

\begin{itemize}
	\item The time needed for building the index (or database in case of SQL and graph database)
	\item Sum of the sizes of the index and data representation
	\item Time needed for obtaining the candidate set
	\item Time needed for verification of the candidate set
	\item Hit ratio of the candidate set
	\item Total time for query execution
\end{itemize}

Candidate set related numbers are not applicable for SQL and graph database since we do not create a candidate set during the query execution.\\

We have measured the results on laptop Dell Inspiron 15 7000 with Intel(R) Core(TM) i7-6700HQ CPU processor with frequency of 2.60GHz and 16GB of RAM with installed Windows 10 operating system.\\

As a data set we have used the first 100 000 compounds of the \textit{ChEMBL} database \cite{chembl} release 24. We wanted to have large enough data set to have as precise results as possible. On the other hand, most of the experiments are computed just in memory and therefore the data set cannot be larger. We did not find any order pattern in which the compounds are placed into the database and therefore we assume that this order is random.\\

Here are the statistics of the used data set:

\begin{itemize}
	\item \textbf{Vertex count of the smallest compound:} 1
	\item \textbf{Vertex count of the largest compound:} 548
	\item \textbf{Average vertex count:} 28
	\item \textbf{Average edge count:} 30 
	\item \textbf{Number of vertex labels:} 18 
	\item \textbf{Number of edge labels:} 4 
\end{itemize}

For query testing, we have created four sets of queries with sizes of 4, 8, 16 and 24 vertices respectively. Each set contains 10 different queries defined in the SMILES language \cite{smiles}. Queries are listed in attachment \nameref{queries}. All measured values are listed in attachment \nameref{measurements}.\\

At the end of this chapter we summarize the results and use them to prove or disprove the hypotheses formulated in Section \ref{hypotheses}.

\section{Index Building Time}

We define index building time as the time difference between the time when the chemical database is loaded into the memory and the time the index/database is ready to execute queries.\\

In case of \textit{GraphGrepSX}, \textit{GString} and \textit{GIRAS} it means the index building itself, in case of SQL and graph database it means the set of API calls to upload the data into the database. The complete results are provided in Figure \ref{fig:indextime}. For better graph scale we also present results with excluded SQL database results in graph at Figure \ref{fig:indextimenosql}\\

\begin{figure}[h]
	\centering
	\includegraphics[width=1\textwidth]{../img/indexBuildingTime.pdf}
	\caption{Index building time}
	\label{fig:indextime}
\end{figure}

\begin{figure}[h]
	\centering
	\includegraphics[width=1\textwidth]{../img/indexBuildingTimeNoSQL.pdf}
	\caption{Index building time without SQL database}
	\label{fig:indextimenosql}
\end{figure}

We may see that it is significantly slower to create the database from scratch than create just an index as it is in case of \textit{GraphGrepSX} and \textit{GString}. The results of database methods are quite convincing - SQL database creation time is 50 times slower compared to PGX. This is quite an expected result since PGX does work only in memory contrary to the SQL database which writes all the data to the disk.\\

In other two methods the difference is not so significant. \textit{GraphGrepSX} is two times slower than \textit{GString}. There might be two reasons for this observation. The first one is that for \textit{GString} we have used smaller parameter $l$ compared to \textit{GraphGrepSX}. The other explanation might be that it is worth to spend some time in condensation process, because it significantly reduces the number of distinct paths in the graph and therefore it makes the index building process faster.\\

We are not presenting the results for \textit{GIRAS}. The reason is that we were not able to get the results in a reasonable time. Even for 10 000 compounds we did not get the built index even after two days of computation. The reason is that the data set contains small structures which are substructures of many others and therefore there are not present any rare subgraphs.\\

After two days of computation for 10 000 compounds we stopped at the moment where we were missing indexing of 39 compounds and the currently searched support level was 600. In other words, these 39 compounds do not contain any subgraph (of maximal size of 8 vertices) which is rare enough to be a part of less than or equal to 600 compounds in the data set.\\

Just for verification, we tested \textit{GIRAS} on small datasets (hundreds of compounds) and the computation has finished in a reasonable time (several hours) and with expected results.\\

Because we were not able to build the \textit{GIRAS} index for our data set, we do not present the results of other metrics for this method since we had no way to measure them.

\section{Index and Data Size}
Since it is tricky to measure just the index size (and in a graph database this term does not even make sense) we have decided to measure the whole amount of memory needed for a particular method.\\

In case of \textit{GraphGrepSX} and \textit{GString} it is the memory used by the running process after the index is built and after triggered garbage collection.\\

In case of the SQL database we are querying the size of the index structure and \textbf{BONDS} table itself. The query looks as following:\\

\noindent\textbf{SELECT sum(bytes)/1024/1024 as "SIZE in MB"}\\
\phantom{x}\hspace{3ex} \textbf{FROM dba\textunderscore segments}\\
\phantom{x}\hspace{3ex} \textbf{WHERE segment\textunderscore name='BONDS/INDEX\textunderscore NAME'}\\

In case of graph database we use the \textit{getMemoryMb} method which is offered by the Java API of PGX graph representation.\\

The results are provided in Figure \ref{fig:indexsize}. What we found as an interesting observation is the size of the \textit{GString}'s index. After we saw these numbers we started to investigate what is the reason. It turned out that the premise that using the condensed graph to reduce the number od different paths is not valid. We have found out that the built index on the tested database contains more than one a half million nodes in the \textit{GString} index tree. The root node itself has almost 150 children, i.e. there are almost 150 different node types. This is a huge number compared to \textit{GraphGrepSX} which contains only 21 vertex node types.\\

\begin{figure}[h]
	\centering
	\includegraphics[width=1\textwidth]{../img/indexSize.pdf}
	\caption{Index size}
	\label{fig:indexsize}
\end{figure}

Other results are quite expected. The reason why PGX data representation is significantly smaller compared to SQL database is that PGX does not build any indices. The amount of memory consumed by SQL table representation (without the indices) is about the same as for PGX.

\section{Candidate Set Creation Time}
This metric is meaningful only for \textit{GraphGrepSX} and \textit{GString}. As we described in Chapter \ref{experimental}, the candidate set concept is not applicable to SQL and graph databases.\\

By candidate set creation we mean the time difference between the time when a query is executed and the time when we finish the index utilization for the particular query.\\

The results are provided in Figure \ref{fig:candidateset}. We can see that \textit{GString} is significantly slower compared to \textit{GraphGrepSX}. This is most probably because of the significantly bigger index size.\\

\begin{figure}[h]
	\centering
	\includegraphics[width=1\textwidth]{../img/candidateSet.pdf}
	\caption{Candidate set creation time}
	\label{fig:candidateset}
\end{figure}

The other interesting fact for \textit{GString} is that due to the "stars versus paths" issue described in Section \ref{gstring} it is almost impossible to get meaningful results for large queries and therefore the candidate sets are cut down to almost empty sets.

\section{Verification Time}

Verification time does have two different meanings in this context. For methods where we create the candidate set, we understand verification time as time needed to verify the candidate set.\\

In case of SQL and graph databases, where we do not work with the candidate set concept, we understand verification time as the time needed for executing the query since we need to verify every single record in the database.\\

The results are provided in Figure \ref{fig:verification} in which we can observe several interesting outcomes.\\

\begin{figure}[h]
	\centering
	\includegraphics[width=1\textwidth]{../img/verification.png}
	\caption{Verification time}
	\label{fig:verification}
\end{figure}

At first, we can be surprised by very low numbers for verification time in case of \textit{GString}. This is caused by a significantly smaller candidate set compared to \textit{GraphGrepSX}. On the other hand, the candidate set is not smaller because of better pruning ability of \textit{GString} index, but because of the fact that \textit{GString} invalidates even the results which are valid for other methods. This was described in detail in Section \ref{gstring}.\\

The other interesting observation are the values for PGX. Although, the values for queries of sizes 4 and 8 are very good (even better than for \textit{GraphGrepSX} which is indexed) we have found out that for queries with the size bigger than 14 it is barely usable.\\

Also, we have tried to test it even on database with 1 graph with 2 vertices and 1 edge between them. We would expect that any query will be executed quickly because there is not much to compute. However, we have found that even on this small graph, big queries are very slow and the complexity grows exponentially, while query with 12 vertices took 46 seconds and query with 14 vertices took 50 minutes. Even after 3 hours of computation we were not able to get results for query with 16 vertices.\\

It seems that PGX spends a lot of time on PGQL query parsing and on creation of execution plan. We were unsure whether we did anything wrong. Luckily, the author of this thesis consulted this issue with a member of the Oracle PGX team, who confirmed that the query structure is correct and that it takes an enormous time even on Oracle internal infrastructure. We may then doubt how valid results are described in presentation \cite{pgx-neo4j} where very promising numbers even for large queries are presented.

\section{Hit Ratio of Candidate Set}
This section is only applicable for methods which create the candidate set. The metric is defined as a ratio between the candidate set size and the result set size. It measures the quality of the index, i.e. the higher the ratio is, the better results are obtained from the index.\\

The results are available in Figure \ref{fig:hitratio}. We can see that for \textit{GraphGrepSX} the efficiency of its index decreases with the query size. This is natural since the \textit{GraphGrepSX}'s index describes only paths of length up to 6. Therefore, it is expectable that with growing size of query, the accuracy will decrease, since the indexed paths cover a smaller portion of the query.\\

\begin{figure}[h]
	\centering
	\includegraphics[width=1\textwidth]{../img/hitRatio.pdf}
	\caption{Candidate set hit ratio}
	\label{fig:hitratio}
\end{figure}

On the other hand, even queries of size 24 are not big enough to overflow the capacity of \textit{GString}. The condensed graph does not contain paths longer than 5 in these cases and therefore we would expect more or less constant hit ratio for all query sizes which matches the actual results. However, we can see that the hit ratio is significantly smaller compared to \textit{GraphGrepSX}.

\section{Query Execution Time}
In this section we describe the time results for the whole query process. It can be defined as a sum of the candidate set creation time and verification time. Note that in case of SQL and graph databases this is equal to the verification time. For the end user, this is probably the most crucial metric.\\

The results are provided in Figure \ref{fig:querytime}. The first thing we can observe is that this graph is not much different from the one in Figure \ref{fig:verification}. In other words, the time for obtaining a candidate set plays only very minor role in the total query time.\\

\begin{figure}[h]
	\centering
	\includegraphics[width=1\textwidth]{../img/queryTime.png}
	\caption{Query time}
	\label{fig:querytime}
\end{figure}

The very good performance of \textit{GString} is a result of the fact that the result set is smaller compared to other methods. This might be confusing and a user of such method has to be aware of its limitations. On the other hand, if the user knows what are the \textit{GString} restrictions, it may be a very efficient way of querying.\\

In case of small queries, the best choice seems to be PGX. The implementation is very straight-forward and most of the work is very intuitive. Also, the implementation of PGX handler is very easily improvable to work with even much bigger data sets which cannot fit into the memory.\\

In case of the SQL database, we are quite surprised that it is a viable solution. The difference in performance times to other methods is not that significant as we would expect. Also, SQL solution is the only one which does not have to fit into memory as it is.\\

Although all other methods have their own benefits, \textit{GraphGrepSX} seems to be an overall winner. It is quite simple to implement, it has the best overall performance and reasonable index size as well as its build time.

\section{Hypotheses Results}
In this section we summarize the results of the hypotheses formulated in Section \ref{hypotheses} in Table \ref{hypothesisresults}.

\begin{table}[h]
	\centering
	\renewcommand{\arraystretch}{2.5}
	\setlength{\arrayrulewidth}{0.5mm
	}
	\begin{tabular}[h!]{|p{2cm}|p{2cm}|p{9cm}|}
		\hline
		\rowcolor{lightgray}
		Hypothesis & Result & Comments\\ \hline
		H1.1 & False & Index of \textit{GString} is significantly larger. The number of distinct nodes in case of \textit{GString} is much bigger compared to \textit{GraphGrapSX} on a real-world data set.\\ \hline
		H1.2 & Uncertain & The performance of \textit{GString} is indeed better compared to \textit{GraphGrepSX}. On the other hand, the main reason is a smaller answer set because the rules for candidate set creation are too restrictive in some cases.\\ \hline
		H2.1 & Cannot be verified & We were not able to build the \textit{GIRAS} index in a reasonable time.\\ \hline
		H2.2 & True & We were not able to build the \textit{GIRAS} index in a reasonable time even for the one tenth of the tested data set size.\\ \hline
		H3.1 & True & In general, both \textit{GraphGrepSX} and \textit{GString} perform better than SQL and PGX approaches. On the other hand, for small queries, the PGX is slightly faster. Moreover, in case of the SQL database we did expect much worse results.\\ \hline
		H3.2 & Uncertain & The hypothesis is definitely valid for small queries, in which case the performance difference is enormous. On the other hand, for larger queries PGX starts to be barely usable due to the issues with PGQL query parsing.\\ \hline
	\end{tabular}
	\caption{Hypotheses results}
	\label{hypothesisresults}
\end{table}

\chapter*{Conclusion}
\addcontentsline{toc}{chapter}{Conclusion}

Possible continuation o research:

 - What about combining the indices with graph databases? 

 - Do some performance comparison of commercially used solutions with the other solutions.

 


%%% Bibliography
\include{bibliography}

%%% Figures used in the thesis (consider if this is needed)
\listoffigures

%%% Tables used in the thesis (consider if this is needed)
%%% In mathematical theses, it could be better to move the list of tables to the beginning of the thesis.
\listoftables

%%% Abbreviations used in the thesis, if any, including their explanation
%%% In mathematical theses, it could be better to move the list of abbreviations to the beginning of the thesis.
\chapwithtoc{List of Abbreviations}

\begin{itemize}
	\item DFS - Depth-First search
	\item DAG - Directed Acyclic Graph
	\item DBMS - DataBase Management System
	\item SQL - Structured Query Language
	\item BFS - Breadth-First Search
	\item REST - REpresentational State Transfer
	\item CDK - Chemistry Development Kit
	\item SSSR - Smallest Set of Smallest Rings
	\item MCB - Minimum Cycle Basis
	\item API - Application Programming Interface
	\item JDBC - Java DataBase Connectivity
	\item PGX - Parallel Graph analytiX
	\item PGQL - Property Graph Query Language
	\item ACID - Atomicity, Consistency, Isolation, Durability
	\item CPU - Central Processing Unit
	\item RAM - Random Access Memory
\end{itemize}

%%% Attachments to the master thesis, if any. Each attachment must be
%%% referred to at least once from the text of the thesis. Attachments
%%% are numbered.
%%%
%%% The printed version should preferably contain attachments, which can be
%%% read (additional tables and charts, supplementary text, examples of
%%% program output, etc.). The electronic version is more suited for attachments
%%% which will likely be used in an electronic form rather than read (program
%%% source code, data files, interactive charts, etc.). Electronic attachments
%%% should be uploaded to SIS and optionally also included in the thesis on a~CD/DVD.
\chapwithtoc{Attachments}
\section*{Set of tested queries} \label{queries}
\subsection*{Queries of size 4}
\begin{enumerate}
	\item c:c:c:c
	\item C=NCC
	\item S(=O)(=O)C
	\item C=CC=C
	\item N=CCC
	\item CN=CO
	\item SCCC
	\item c:n:c:c
	\item CSCC
	\item CSC=C
\end{enumerate}

\subsection*{Queries of size 8}
\begin{enumerate}
	\item COc1ccccc1
	\item CCCCCCCC
	\item c:c:c:c:c:c:c:c
	\item S(CCC)(CCC)(C)
	\item C(=O)NCCCCC
	\item c1cc(C(=O))ccc1
	\item CCC(=O)C(=O)NC
	\item c2c(Cl)cccc2Cl
	\item C(C)(C)C(C)(C)C(=O)
	\item S(CNCC)(C=O)(C)
\end{enumerate}

\newpage
\subsection*{Queries of size 16}
\begin{enumerate}
	\item CNC(=O)c1cc(C(=O)CCCC)ccc1
	\item CCCCCCCCCCCCCCCC
	\item c:c:c:c:c:c:c:c:c:c:c:c:c:c:c:c
	\item c1cccc2c1ccc3c2cccc3CC
	\item S(CC)(=O)(=O)Nc1cc(CCCC)ccc1
	\item C(C)(C)C(C)(C)C(C)(C)C(C)(CCCCC)
	\item C(O)CCCC(=C)CCCc1ccccc1
	\item CCCCCNCCCCCCCNCC
	\item c1ccccc1CCc2cc(CC)ccc2
	\item Oc1cccc2Cc3ccccc3C(=O)c12
\end{enumerate}

\subsection*{Queries of size 16}
\begin{enumerate}
	\item C(C)NC(=O)C(NC(=O)OC)CCCCNC(=O)c1ccccc1
	\item CCCCCCCCCCCCCCCCCCCCCCCC
	\item c:c:c:c:c:c:c:c:c:c:c:c:c:c:c:c:c:c:c:c:c:c:c:c
	\item c1cccc2c1ccc3c2ccc4c3ccc5c4cccc5CC
	\item S(C)(=O)(=O)Nc1cc(C(c2c(=O)oc(CC)cc2)CCC)ccc1
	\item C(C)(C)C(C)(C)CCCC(C(NCCCCC)=O)(CCCCCC)
	\item C(O)C(O)C(O)C(O)CCCC(=C)C(O)C(C)Cc1ccccc1
	\item CCCCCNCCCCCCCNCCCCCCCCCC
	\item c1ccccc1CCc2cc(CCCNc3ccccc3)ccc2
	\item Oc1cccc2Cc3ccc(Cc4ccccc4)c(O)c3C(=O)c12
\end{enumerate}

\newpage
\section*{Results of measurements} \label{measurements}
\subsection*{Index building}

\begin{figure}[h]
	\centering
	\includegraphics[width=0.75\textwidth]{../img/indexBuilding.png}
\end{figure}

\newpage
\subsection*{Queries of size 4}

\begin{figure}[h]
	\centering
	\includegraphics[width=1\textwidth]{../img/q4.png}
\end{figure}

\newpage
\subsection*{Queries of size 8}

\begin{figure}[h]
	\centering
	\includegraphics[width=1\textwidth]{../img/q8.png}
\end{figure}

\newpage
\subsection*{Queries of size 16}

\begin{figure}[h]
	\centering
	\includegraphics[width=1\textwidth]{../img/q16.png}
\end{figure}

\newpage
\subsection*{Queries of size 24}

\begin{figure}[h]
	\centering
	\includegraphics[width=1\textwidth]{../img/q24.png}
\end{figure}

\openright
\end{document}
