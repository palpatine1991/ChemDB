\chapter*{Introduction}
\addcontentsline{toc}{chapter}{Introduction}

Querying is the essential utility of each database. The same applies to chemical databases whose size is growing rapidly and therefore there is a pressure on efficiency of the querying process. Since the chemical compounds are typically represented as graphs, the most common queries over chemical databases are exact match query, shortest path search, similarity search and substructure search which are usually used in graph databases. The latter will be the main point of interest in this thesis.\\

The goal of this thesis is to compare the efficiency of querying methods which have been already proposed in other papers. This includes comparison of algorithms and also the utilization of native query mechanisms of both graph and relational databases.\\

We will focus on the general performance of these approaches as well as on the particular cases where some approach might be better than others.


\section*{Structure of the Thesis}
\addcontentsline{toc}{section}{Structure of the Thesis}

This thesis is divided into three main parts. In the first part we will analyze the subgraph querying problem. We will define the basic terms, list the algorithms for resolving subgraph isomorphism problem and most importantly we will analyze the related work. \\

In the second part several hypotheses will be uttered. For their verification the author’s experimental work will be used. These experiments will be described and the issues found out during the implementation will be explored.\\

The last part of the thesis will cover the results of experimental work and the comparison with results of related researches. We will comment on the findings and propose some directions in possible following research.
